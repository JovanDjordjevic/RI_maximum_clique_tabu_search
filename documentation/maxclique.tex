\documentclass[12pt]{article}
\usepackage[utf8]{inputenc}
\usepackage[english,serbian]{babel}
\usepackage{hyperref}
\hypersetup{
colorlinks,
linkcolor=blue,
urlcolor=blue
}

\title{Maximum clique korišćenjem Tabu pretrage \vspace{0.4cm}\\
\large Projekat iz Računarske inteligencije \\ Matematički fakultet \\ Univerzitet u Beogradu  \vspace{1cm} \\}
\author{Jovan Đorđević\\
\href{mailto:mi17164@alas.matf.bg.ac.rs}{mi17164@alas.matf.bg.ac.rs} \\
Tamara Tomić\\
\href{mailto:mi17122@alas.matf.bg.ac.rs}{mi17122@alas.matf.bg.ac.rs} \vspace{0.5cm} \\
}
\date{Decembar 2021}

\begin{document}

\maketitle

\newpage

\renewcommand*\contentsname{Sadržaj}
\tableofcontents

\newpage

\section{Opis problema} \\
Neka je G = (V, E) neusmeren graf, gde je $V = \{1,...,n\}$ skup čvorova i $ E \subset V $ x $ V $ skup grana grafa. Klik C grafa G je podskup V takav da su svaka dva čvora povezana , odnosno da važi $\forall u, v \in C, \hspace{0.1cm} $\{u, v\}$ \hspace{0.1cm} \in E $. Klik C je maksimalan ako nije sadržan ni u jednom drugom kliku, odnosno ako je njegova kardinalnost veća od kardinalnosti svih drugih klikova. Pronalaženje maksimalnog klika je NP-težak problem, koji ima primenu u rešavanju mnogih svakodnevnih problema - teorija klasifikacije, dijagnostika kvara, biološke analize, analize klastera, izbor projekta, itd. Problem pronalaska maksimalnog klika je ekvivalentat problemu maksimalnog nezavisnog skupa i uskoro povezan sa problemom bojenja grafa.

\section{Ulazni podaci i struktura Graf}

Graf smo implementirali preko matrice povezanosti. Matrica je dimenzije N x N gde je N broj cvorova + 1. Ako u grafu postoji grana izmedju cvorova 'i' i 'j',   i,j \in V \end, vrednost matrice m[i][j] ce biti jednaka 1, u suprotnom 0, tj: \\
$$
adjacentMatrix[i][j] =
\left\{
	\begin{array}{ll}
		1,  & \mbox{if } (i, j) \in E \\
		0,  & \mbox{if } (i, j) \notin E \\
	\end{array}
\right.
$$ \\
Omogućili smo lepo tabelirano prikazivanje matrice povezanosti grafa preoprerećenjem funkcije $ __str__ $ iz Graph klase. Potrebno je instalirati 'tabulate' pomoću komande : \textbf{pip install tabulate}. \\
Test instance na kojima će biti testiran naš algoritam mogu se prezuzeti sa sledeće adrese: \href{https://networkrepository.com/dimacs.php}{https://networkrepository.com/dimacs.php}. \\
Zadati grafovi se tretiraju kao neusmereni grafovi bez težina. U prvom redu svakog fajla navedene su informacije o broju čvorova u grafu, a u svakom narednom redu navedena je jedna grana. Najmanji graf u ovoj kolekciji ima 28 čvorova i 420 grana, dok najveći imaju oko 3 do 4 hiljade cvorova i oko 4 do 6 miliona grana. Ovaj set test instanci je izabran kako bi što je moguće blize uporedili nas rezultat sa rezultatom iz rada [1]. Bitno je prilikom analize rezultata imati na umu razlike u implementaciji. U originalnom radu [1], AMTS algoritam implementiran je u programskom jeziku C, kompajliran pomocu GNU GCC i pokrenut na racunaru sa 2GB RAM memorije i 2.61 GHz procesorom.
Naš algoritma implementiran je u sporijem programskom jeziku (Python), ali pokrenut na dosta bržem računaru (32GB RAM, 5.0 GHZ CPU).



\section{Tabu pretraga - uopšteno}
Tabu pretraga je metaheuristika koja spada u algoritme lokalne pretrage i koristi se za rešavanje problema kombinatorne optimizacije. Cilj ove vrste pretrage je da izbegne zaglavljivanje u lokalnim minimumima, što se postiže korišćenjem tabua (zabranjenih stanja). Odražava se tzv. tabu lista, tj. lista stanja u kojime je pretraga već bila i u koja se više neće vraćati. Dozvoljavaju se potezi koji pogoršavaju vrednost funkcije, pod uslovom da nema boljih dozvoljenih poteza. Ova dva pristupa omogućavaju da se napisti lokalni mininum.

\newpage
\addcontentsline{toc}{section}{Literatura}
\begin{thebibliography}{9}

\bibitem{} 
{Qinghua Wu, Jin-Kao Hao -
\textit An adaptive multistart tabu search approach to solve the maximum clique problem}

\end{thebibliography}

\end{document}
